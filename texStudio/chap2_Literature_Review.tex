\chapter{Literature Review}

\section{Introduction}

In 2019, the U.S. Department of Transportation reported more than 36,000 fatalities and 4.4 million critically injured individuals due to accidents involving vehicles \cite{Vehicle-Fatalities2020, Wagner2020}, making motor accidents the third leading cause of death in the United States \cite{Leading-Causes-Death2021}. Among these reports, 90\% result from human error (i.e., the improper reaction to impending danger) \cite{Canada-Driving-Stats2021}. As urbanization continues to grow, so does the expected number of drivers on the road, ultimately increasing traffic congestion, reducing the space between vehicles, shortening the window of time drivers have to:
\begin{itemize} 
	\item Assess a situation;
	
	\item Evaluate their options;  and

	\item React safely.
\end{itemize} 
These factors increase delays for EVs and the risk of EV-involved accidents during an emergence call \cite{Vehicles-on-Road2020}. Additionally, traffic congestion resulting from inefficient avoidance of EVs and areas surrounding an emergency site damages our environment from the emissions of each car \cite{Idling-Impacts2016, Drop2021}.
In this paper, we'll focus on leveraging technology to mitigate three significant biological limitations, as outline below, related to human error and are contributing factors to EV-involved accidents:

\begin{itemize}
	\item Perception;
	
	\item Communication; and
	
	\item Cognitive processing.
\end{itemize}

We define these three factors below:
\begin{itemize}
	\item Perception is the ability to sense and identify danger. While humans rely on various biological senses to navigate the world, only a select few senses provide relevant data while operating a vehicle, such as sound and sight \cite{Sukru2020}. Drivers generally only use sound to identify honking and sirens; they filter out most other noises. Sight is the most used sense by drivers, but every vehicle has an array of blind spots, and many threats live outside their Line-of-Sight (\acrshort{LoS}), usually obstructed by other vehicles, buildings, trees, and poor weather conditions \cite{Sukru2020};
	
	\item Communication is the ability to perceive neighbouring drivers' intentions unambiguously and explicitly express your intentions to those around you. The status quo for car manufacturers is to equip commercial vehicles with external indicators for communication purposes, including a monotone horn, signal lights, and brake lights. But drivers must consciously and manually use the indicators, and with the use of these indicators varying between cultures and individuals, there is too much room for error or misinterpretation;
	
	\item Cognitive processing is the ability to understand your immediate surroundings and plan strategies for avoiding or preventing dangerous situations by collecting and accessing environmental data. Drivers already have high cognitive workloads given many factors such as unfamiliar roads, poor weather conditions, and multitasking, to name a few. Even in optimal conditions, drivers often only have a few seconds to react to threats given the high speeds they travel at, and the decisions they make tend to be ill-informed guesses that often lead to accidents \cite{Buchenscheit2009}.
\end{itemize}


\section{Intelligent Transport Systems}

Intelligent Transport Systems (\acrshort{ITS}) are advanced systems improving efficiency and safety of various transport-related situations, enabling drivers to make better informed, safer, and more coordinated decisions \cite{Huang2009, Khare2020, Beneoks2019, Drop2021}. We will refer to vehicles using \acrshort{ITS} applications as connected vehicles (CVs).

There are many mediums for which CVs can communicate with each other and to infrastructure. The two leading approaches are:
\begin{itemize}
	\item Radio; and

	\item Cellular.
\end{itemize}

Radio communication is an area that has seen rapid growth, especially in the Dedicated Short-Range Communication (\acrshort{DSRC}) protocol, also known as IEEE 802.11p. \acrshort{DSRC} periodically broadcasts messages every 300 milliseconds, where each message contains the vehicle's speed, acceleration, location, and heading \cite{Hafeez2013, Khare2020}. At the time of this writing, no commercial vehicles are using \acrshort{DSRC}. On the contrary, cellular technology has seen massive adoption in the automotive industry; Tesla Motors adds built-in LTE modems to all their vehicles \cite{Orange-Telsa2014, TelsaConnectivity2021}. These modems enable interactive navigation services, including live traffic visualization, satellite-view maps, and navigation \cite{Orange-Telsa2014, TelsaConnectivity2021}. Because cellular technology utilizes existing infrastructure such as cellular towers, the adoption and scaling costs are significantly smaller than \acrshort{DSRC}. For that reason, cellular is the communication technology of choice for this research.

Many countries have invested in standardizing traffic management communication infrastructure throughout the last decade, hoping to increase the demand for CVs \cite{Huang2009}. Nevertheless, despite the promising results in the literature, CVs are only now emerging on the market and are not yet highly available, so their safety and assistive features are yet to be fully realized \cite{Bahaaldin2017, Ahmed2017}. However, with the recent growth of popularity surrounding autonomous vehicles over the last decade, the growing demand for vehicular safety features, and stringent government rules for improved traffic management, more comprehensive implementation of CVs is inevitable \cite{Sukru2020, Global-V2X-Market2020, Lyu2020}. In the following sections, we'll examine the literature on two critical areas that \acrshort{ITS} applications that have shown promising results:
\begin{itemize}
	\item Accident Prevention; and

	\item Traffic Flow Optimization
\end{itemize}

\section{\acrshort{ITS} in Accident Prevention}
Civilian drivers rely too heavily on \acrshort{LoS} to perceive their surroundings, often having difficulty seeing or sensing obstacles obstructed by other vehicles, buildings, trees, or weather conditions \cite{Sukru2020}. Even with the technological advances offered by modern cars such as \gls{LIDAR}, \gls{RADAR}, and cameras, these sensors fundamentally rely on \acrshort{LoS}, thus showing minor improvement for long-distance threat detection \cite{Sukru2020}. This review highlights two problem areas related to Non-Line-of-Sight (NLoS) vehicle sensing and guided driving.


\subsection{NLoS Vehicle Sensing}
The root of many of the accidents reported in the literature stems from drivers' obstructed vision, either due to blind spots, poor weather conditions, or any number of other causes. NLoS vehicle sensing enables CVs to sense each other despite obstacles that would otherwise hide their presence \cite{Subedi2020}.

One study by \cite{Buchenscheit2009} focused on the safety of EVs. Many emergency responders reported driving more than 5 million miles a year and often operated under heavy visual, mental, and cognitive workloads, potentially driving at high speeds through difficult traffic and weather conditions \cite{Buchenscheit2009}. EVs traditionally rely on sirens and lights to gain nearby civilian drivers' attention but have been proven inefficient at negotiating congested urban traffic. The warning is often recognized too late and conveys only the general location of an EV when outside the \acrshort{LoS} of the civilian driver \cite{Buchenscheit2009}. Without communicating intent and context, civilian drivers will continue making poorly-informed decisions that could lead to further accidents. 

Multiple studies leveraged \gls{V2I} communication by installing RSUs near major intersections and using a centralized server to manage traffic data and disseminate information via \acrshort{DSRC} \cite{Buchenscheit2009, Huang2009}. Major drawbacks to this approach are its dependency on the high number of RSUs needed throughout a city to ensure high coverage, and the short-range of \acrshort{DSRC} means CVs are not guaranteed to be communicating in real-time. Despite these limitations, the results from these studies support that using a centralized server to suggest vehicular navigation information to CVs will provide civilian drivers with enough context to make safer and better-informed decisions. Our research adopts a similar functioning centralized server to negotiate traffic flow, focusing on a \acrshort{V2I} communication approach. Still, we will use cellular communication instead of \acrshort{DSRC}, given cellular's dominating market hold over \acrshort{DSRC}.


\subsection{Guided Driving}
Lane changes are among the most fundamental processes for drivers. However, they account for about 5\% of traffic accidents \cite{Ni2020} and 10\% traffic congestion \cite{Ni2020}. Among these reported accidents, human error was a factor for 75\% \cite{Ni2020}. CVs can suggest faster, safer, and more predictable lane-changing planning and speed control strategies than unaided drivers. 

There are many studies on cooperative lane-changing algorithms. One study proposes a multi-vehicle strategy in which the decision-making control is decentralized \cite{Ni2020}. This approach creates a more comfortable experience for the involved drivers than unaided lane changes while simultaneously increasing traffic flow and road safety due to offloading the calculation of optimal decisions to an \acrshort{ITS}. Unfortunately, the research failed to consider the delays in \acrshort{DSRC} communication and systems response times. Additionally, this approach requires a high penetration rate of CVs. 

The research by \cite{Ahmed2017} developed a \acrshort{DSRC}-based freeway merging assistant system. The authors tested against various lane-merging scenarios using three CVs and smartphones as a \acrshort{GUI} for displaying advisory messages. Although lane-merging is not the focus of our research, and the tested scenarios were basic in design, they were performed in an uncontrollable environment, demonstrating that real-world route guidance systems are feasible and effective even in complex environments.

In the third study, authors \cite{Bahaaldin2017} focused on improving and maintaining traffic flow during emergency evacuations. The experimenters varied the penetration rate of CVs from zero percent (i.e., base scenario) to thirty percent (i.e., the predicted rate by 2018). The algorithm suggested which lane and speed to maintain based on neighbouring CVs' traffic flow data. The study's limitations were in the assumptions that drivers of CVs would obey every suggestion given by the system. The study results demonstrated that increasing the penetration rates of CVs alone was not enough to mitigate traffic congestion but instead should be used in tandem with other techniques or technologies. 


\section{\acrshort{ITS} in Traffic Flow Optimization}
\subsection{Route Guidance}
We define \gls{Route Guidance} as the problem of computing an optimal route (by some criteria such as distance or time) between an origin and a destination and adapting to real-time traffic updates guiding the driver on how best to avoid congested traffic. Given the time-sensitive nature of emergencies, EVs need to minimize arrival times by maintaining high speeds and avoiding unnecessary delays. In addition to the high accident risk, civilian drivers' ill-informed decisions also delay EVs. For example, in traffic jams, drivers often do not know how and where to form a suitable corridor to let the EVs through as lights and sirens fail to convey intent \cite{Buchenscheit2009}.

The study by \cite{Rizvi2007} uses real-time traffic information to avoid congested road sections. The proposed model takes the approach to minimize prerequisite infrastructure by creating a \acrshort{VANET} from CVs instead of relying on RSUs. These suggested detours would guide drivers away from primary to secondary roads, increasing the overall commute distance and encouraging safer and smoother commutes. The shortfall of this approach is that the commuting time for the detoured vehicles will increase given the potentially longer or less direct route on the secondary roads. That said, our research will utilize this technique only for civilian drivers as their arrival time is of less importance than those of EVs.

In the second study by \cite{Huang2009}, a centralized server controls all traffic lights and traffic information. It is also responsible for computing the shortest-time plan and alternative routes, calculated with the \gls{A* algorithm} based on distance and average expected speeds for EVs. The \gls{A* algorithm} is a best-first graph search algorithm that can find the shortest path. The authors used the relationship between the distance from a given location along the vehicle's route and its average velocity as the heuristic function used within this algorithm. The first issue addressed is computing the fastest path from the source to the event (destination) for the EVs and adjusting this route based on real-time traffic. The second challenge is to disseminate the warning messages to nearby CVs along the EV's route, advising them to move or stay put to avoid route collisions.

\subsection{Traffic Light Preemption}
Many factors contribute to the increasing traffic congestion in urban areas, but intersection traffic lights play a significant role in regulating traffic flow. Traditional approaches use inefficient timer-based decision logic, merely toggling the right-of-way (i.e., green light) signal between the competing directions at a fixed interval. Unfortunately, traffic flow for most of the time is not symmetric, resulting in unnecessary traffic congestion. One study implemented \acrshort{DSRC}-actuated traffic lights using off-the-shelf hardware and software to reduce traffic congestion by prioritizing CVs \cite{Tonguz2020}. The significant reduction in traffic congestion despite a low CV penetration rate makes this approach easily deployable. Another approach makes use of a centralized server to preempt all traffic lights (i.e., displaying a red light in all directions) when an EV is approaching \cite{Vlad2008}. The intent is to stop all traffic such that no driver will collide with the EV.
 
Consequently, stopped traffic means greater congestion and may cause more chaos in nearby roadways. Similarly, another approach entails giving the direction of an approaching EV the right-of-way (i.e., displaying a green light) such that vehicles can move and clear a path \cite{Buchenscheit2009}. This approach does not warn civilian drivers of the nearby EV, and it also relies heavily on the presence of modified traffic lights to control traffic flow.

\section{Conclusion}
This chapter examined the applications of CVs and highlighted the multitude of advantages and limitations of daily use and emergencies. We explored existing literature that leverages CV applications to combat the perception, communication and cognitive processing issues civilian drivers face in emergencies.