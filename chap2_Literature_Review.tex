\chapter{Literature Review}

\section{Introduction}

In 2019, the U.S. Department of Transportation reported more than 36,000 fatalities and 4.4 million critically injured individuals due to accidents involving vehicles \cite{Vehicle-Fatalities2020}, making motor accidents the third leading cause of death in the United States \cite{Leading-Causes-Death2021}. Among these reports, 90\% result from human error (i.e., the improper reaction to impending danger) \cite{Canada-Driving-Stats2021}. As urbanization continues to grow, so does the expected number of drivers on the road, ultimately increasing traffic congestion, reducing the space between vehicles, shortening the window of time drivers have to assess a situation, evaluating their options, and reacting safely. These factors increase delays and the risk of ER-involved accidents during an emergence call \cite{Vehicles-on-Road2020}. Additionally, traffic congestion due to inefficient avoidance of \acrshort{ER}s and emergency sites drastically damages our environment from the emissions of each car \cite{Idling-Impacts2016, Drop2021}.
Some of the main factors contributing to \acrshort{ER}-involved accidents relating to human error include biological limitations, such as perception, communication and processing, as outlined below:

\begin{itemize}
	\item Perception is the ability to sense and identify emergencies. While humans rely on various biological senses to navigate the world, only a select few provide relevant data while operating a vehicle, such as sound and sight \cite{Sukru2020}. Drivers generally only use sound to identify honking and sirens; they filter out most other noises. Sight is the most used sense by drivers, but every vehicle has an array of blind spots, and many threats live outside their \acrlong{LoS} (\acrshort{LoS}), usually obstructed by other vehicles, buildings, trees, and poor weather conditions \cite{Sukru2020};
	
	\item Communication is the ability to perceive neighbouring drivers' intentions unambiguously and clearly express your intentions. Standard vehicles are equipped with few external indicators, including a monotone horn, signal lights, and brake lights. But the use of these indicators varies between cultures;
	
	\item Processing is the ability to plan strategies for avoiding or preventing dangerous situations by collecting environmental data and assessing the surroundings. Drivers already have potentially high cognitive workloads given many factors such as unfamiliar roads, poor weather conditions, and multitasking, to name a few. Even in optimal conditions, drivers often only have a few seconds to react given the high speeds they travel at, and the decisions they make tend to be ill-informed guesses that often lead to accidents \cite{Buchenscheit2009}.
\end{itemize}


\section{Intelligent Transport Systems}

\acrlong{ITS}s (\acrshort{ITS}) are advanced systems improving efficiency and safety of various transport-related situations, enabling drivers to make better informed, safer, and more coordinated decisions \cite{Huang2009, Khare2020, Beneoks2019, Drop2021}. Vehicles using \acrshort{ITS} applications are hereafter referred to as \textit{connected vehicles}.

One medium for \textit{connected vehicles} to communicate is through the \acrlong{DSRC} (\acrshort{DSRC}) protocol, also known as IEEE 802.11p. \acrshort{DSRC} periodically broadcasts messages every 300 milliseconds, where each message contains the vehicle's speed, acceleration, location, and heading \cite{Hafeez2013, Khare2020}.

Throughout the last decade, many countries have been investing in standardizing traffic management communication infrastructure, hoping to increase the demand for \textit{connected vehicles} \cite{Huang2009}. Nevertheless, despite the promising results in the literature, \textit{connected vehicles} are not yet highly available on the market, and their safety and assistive features are yet to be fully realized \cite{Bahaaldin2017, Ahmed2017}. However, with the recent growth of popularity surrounding autonomous vehicles over the last decade, the growing demand for vehicular safety features and stringent government rules for improved traffic management, more comprehensive implementation of \textit{connected vehicles} is inevitable \cite{Sukru2020, Global-V2X-Market2020, Lyu2020}.


\section{ITS in Accident Prevention}
Civilian drivers rely too heavily on \acrshort{LoS} to perceive their surroundings, often having difficulty seeing or sensing obstacles obstructed by other vehicles, buildings, trees, or weather conditions \cite{Sukru2020}. Even with the technological advances offered by modern cars such as \gls{LIDAR}, \gls{RADAR}, and cameras, these sensors fundamentally rely on \acrshort{LoS}, thus performing poorly in terrible weather conditions \cite{Sukru2020}. This review highlights two problem areas including \acrlong{NLoS} (\acrshort{NLoS}) vehicle sensing and guided driving.


\subsection{NLoS Vehicle Sensing}
In 2018, there were more than 12 million reported car-related accidents in the United States \cite{Wagner2020}, with more than 36,000 involving fatalities \cite{Wagner2020}. The root of many of these accidents stems from the obstructed vision of drivers, either due to blind spots, poor weather conditions, or any number of other causes. \acrshort{NLoS} vehicle sensing enables \textit{connected vehicles} to sense each other despite obstacles that would otherwise hide their presence \cite{Subedi2020}.

One study by \cite{Buchenscheit2009} focused on \acrshort{ER}'s safety. Many \acrshort{ER}s reported driving more than 5 million miles a year and often operated under heavy visual, mental, and cognitive workloads, potentially driving at high speeds through difficult traffic and weather conditions \cite{Buchenscheit2009}. \acrshort{ER}s traditionally rely on sirens and lights to gain nearby civilian drivers' attention but have been proven inefficient at negotiating congested urban traffic. The warning is often recognized too late and conveys only the general location of an \acrshort{ER} when outside the \acrshort{LoS} of the civilian driver \cite{Buchenscheit2009}. Without communicating intent and context, civilian drivers will continue making poorly-informed decisions that could lead to further accidents. Multiple studies leveraged \acrshort{V2I} communication by installing \acrshort{RSU}s near major intersections and using a centralized server to manage traffic data and disseminate information via \acrshort{DSRC} \cite{Buchenscheit2009, Huang2009}. Major drawbacks to this approach are its dependency on the high number of \acrshort{RSU}s needed throughout a city to ensure high coverage and the short-range of \acrshort{DSRC} which means \textit{connected vehicles} are not communicating in real-time. The results from these studies support that communicating vehicular information to \textit{connected vehicles} will provide civilian drivers with enough context to make safer and better-informed decisions.


\subsection{Guided Driving}
Lane changes are among the most fundamental processes for drivers. However, they account for about 5\% of traffic accidents \cite{Ni2020} and 10\% of traffic congestion \cite{Ni2020}. Among these reported accidents, 75\% of them were caused by human error \cite{Ni2020}. With the advances in \textit{connected vehicles}, more optimized lane changing planning and speed control strategies can be suggested to the driver.

There are many studies on cooperative lane changing algorithms. One study proposes a multi-vehicle cooperative lane change strategy in which the decision-making control is decentralized \cite{Ni2020}. This approach creates a more comfortable experience for the involved drivers than unaided lane changes while simultaneously increasing traffic flow and road safety due to offloading the calculation of optimal decisions to an \acrshort{ITS}. Unfortunately, the research failed to consider the perceived errors, delays in communication, and systems response times. Additionally, this approach requires a high penetration rate of \textit{connected vehicles}, which is yet to be seen globally.

A \acrshort{DSRC}-based freeway merging assistant system was developed \cite{Ahmed2017}. Various lane merging scenarios were tested using a smartphone as a \acrshort{GUI} for displaying advisory messages and three \textit{connected vehicles}. Although the tested scenarios were basic, involving only single-hop broadcasting, they were performed in an uncontrollable environment, demonstrating that real-world route guidance systems are feasible and effective even in complex environments.

In the third study, authors \cite{Bahaaldin2017} focused on improving and maintaining traffic flow during emergency evacuations. The experimenters varied the penetration rate of \textit{connected vehicles} from zero-percent (i.e., base scenario) to 30-percent (i.e., the predicted rate by 2018). The algorithm suggested which lane and speed to maintain based on neighbouring \textit{connected vehicles'} traffic flow data. The study results demonstrated that increasing the percent of \textit{connected vehicles} present in an emergency evacuation led to significant traffic delays early into the situation and that the delay benefits would become positive only after approximately 1/3 of the overall time. It also demonstrated that the amount increased is proportional to the penetration rate of \textit{connected vehicles}. The study's limitations were in the assumptions that drivers of \textit{connected vehicles} would obey every suggestion given by the system.


\section{ITS in Road Optimization}
\subsection{Route Guidance}
We define \gls{Route Guidance} as the problem of computing an optimal route (by some criteria such as distance or time) between an origin and a destination and adapting to real-time traffic updates guiding the driver on how best to avoid congested traffic. Given the time-sensitive nature of emergencies, \acrshort{ER}s need to minimize arrival times by maintaining high speeds and avoiding unnecessary delays. In addition to the high accident risk, civilian drivers' ill-informed decisions also delay \acrshort{ER}s. For example, in traffic jams, confused drivers often do not know how and where to form a suitable corridor to let the emergency vehicle through \cite{Buchenscheit2009}.

The study by \cite{Rizvi2007} uses real-time traffic information to avoid congested road sections. The proposed model takes the approach to minimize prerequisite infrastructure by using \textit{connect vehicles} within a \acrshort{VANET} as information servers instead of relying on \acrshort{RSU}s.

In the second study by \cite{Huang2009}, the use of a centralized server controls all traffic lights and traffic information. It is also responsible for computing the shortest-time plan and alternative routes, calculated with the \textit{A* algorithm} based on distance and average expected speeds, for \acrshort{ER}s. The \textit{\gls{A* algorithm}} is a best-first graph search algorithm that can find the shortest path. The authors used the relationship between the distance from a given location along the vehicle's route and the its average velocity as the heuristic function used within this algorithm. The first issue addressed is computing the fastest route from the source to the event (destination) for the \acrshort{ER}s and adjusting this route based on real-time traffic. The second challenge is to disseminate the warning messages to nearby \textit{connected vehicles} along the \acrshort{ER}'s route, advising them to move or stay put to avoid route collisions.

\subsection{Traffic Light Preemption}
Many factors contribute to the increasing traffic congestion in urban areas, but intersection traffic lights play a significant role in regulating traffic flow. Traditional approaches use inefficient timer-based decision logic, merely toggling the right-of-way (i.e., green light) signal between the competing directions at a fixed interval. Unfortunately, traffic flow for most of the time is not symmetric, resulting in unnecessary traffic congestion. One study implemented \acrshort{DSRC}-actuated traffic lights using off-the-shelf hardware and software to reduce traffic congestion by prioritizing \textit{connected vehicles} \cite{Tonguz2020}. The significant reduction in traffic congestion despite a low \textit{connected vehicle} penetration rate, combined with a cost-effective implementation, makes this approach easily deployable. Another approach makes use of a centralized server to preempt all traffic lights (i.e., displaying a red light to all directions) when an \acrshort{ER} is approaching \cite{Vlad2008}. The intent is to stop all traffic such that no driver will collide with the \acrshort{ER}.

Consequently, they cannot control traffic flow without traffic lights and may cause more chaos in nearby roadways. Similarly, another approach entails giving the direction of an approaching \acrshort{ER} the right-of-way (i.e., displaying a green light) such that vehicles can move and clear a path \cite{Buchenscheit2009}. This approach does not warn civilian drivers of the approaching \acrshort{ER}, and it also relies heavily on the presence of modified traffic lights to control traffic flow.

\section{Conclusion}
This chapter highlighted the importance of \textit{connected vehicles} and the multitude of advantages they offer over regular vehicle's daily use and emergencies, as well as their respective limitations. We explored existing literature that leverages \acrshort{ITS} applications and \textit{connected vehicles} to combat the perception, communication and processing issues civilian drivers face in emergencies.