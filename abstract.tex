% Abstract
For emergency responders, every delay in reaching their destination could be the difference between life and death for those involved in accidents, which is why they must reduce their arrival time as much as possible. ERs are legally permitted to exceed speed limitations, run red lights, and ignore stop signs, all of which put them and nearby civilian drivers at risk of accidents. The traditional methods of using emergency lights and sirens are ineffective at providing civilian drivers with enough warning and context to negotiate traffic in urban areas, resulting in congestion, delays, and increased risk. This thesis describes an innovative cloud-based road-traffic risk mediator system and reports on synthetic test case scenarios. The system, built upon IoT-enabled vehicles and cloud computing, aims to minimize the risk of emergency vehicle-involved accidents by mediating traffic volume along ER paths and offloading the maneuvering computation and decision making for drivers of connected vehicles. 
    \linebreak
    \linebreak
    \textbf{\emph{Keywords:}}  IoT, emergency responders, traffic routing, reducing arrival time, reducing traffic volume, improving road safety, cloud computing.
