\chapter{Literature Review}
Given the aging population, it is no surprise that falls among seniors are occurring at an increasing rate \cite{world2018, RN1014, JP2019}. Recognized as a global phenomenon, aging demographics for people 65+ is growing steadily and so is the susceptibility—and consequences of falls \cite{world2018, RN1014, JP2019}.  According to the World Health Organization, nearly 1 in 3 people over the age of 65 falls each year, resulting in 29 million reported falls and admits to medical clinics and hospitals in the USA alone \cite{world2018}. Unlike other age groups, falls in the elderly pose serious health risks due to the increased potential for health-related consequences that transpire after a fall has occurred. Fall-related injuries currently cost \$34 billion (USA), \$14 billion (Canada), with comparable expenditures for other countries \cite{RN1014}.  Despite such a significant problem, there are few implemented solutions that address this global issue.  This review explores general literature review for Falls for Seniors; and Fall Detection Methods.

\section{General Literature Review of Seniors Falling}

The main factors that place elderly individuals at increased risk for falling are: 1) biological, 2) behavioural, 3) environmental, and 4) socio-economic \cite{RN1000}. These groupings are not mutually independent as several risk factors may be applicable to more than one category. Biological risk factors include advanced age, gender, chronic and acute health conditions, cognitive impairments, lower-extremity weakness, physical limitations, gait abnormalities, balance deficits, and altered sensation \cite{RN1000}. Behavioural factors include a history of previous falls and the use of medications especially for persons using four or more prescription drugs \cite{RN1000}. Additional behavioural risk factors include the use of alcohol and other non-medical drugs, fear of falling, poor diet, insufficient exercise, inappropriate footwear, and the use of assistive devices \cite{RN1011}. Environmental risk factors are extremely varied and include poor weather conditions, uneven sidewalk surfaces, slippery floors, cluttered furniture, poor lighting, and use of unsafe equipment such as wheeled beds or chairs \cite{Rub2006}. Socio-economic factors that increase the risk of falling include inadequate housing, low income, lack of social support, and social isolation \cite{RN1001}. Since falls sustained by elderly are generally caused by a combination of risk factors, it is important that health care practitioners consider all four categories when developing management strategies to prevent falls from occurring \cite{RN1000, RN1011, Rub2006}.



\section{Fall Detection Methods}

Fall-detection methods are roughly categorized into four groups: wearable/mobile-device based, radio-wave based, pressure-sensor based, and vision based \cite{RN1012}.

\subsection{Wearable/mobile-device based}
There has been much research into the topic of fall detection alert systems and many different types of implementations exist. Most of the time, wearables are used for detection, especially an accelerometer, sometimes combined with a gyroscope \cite{fda2007, RN999, RN1007}. Virtually all modern smartphones have accelerometers built into them with high accuracy, sensitivity and specificity \cite{RN1034}. Thresholds have been identified based on empirical studies from the accelerometer and gyroscope fall data, so a fall will be triggered when specific conditions are met \cite{RN1034}. The drawbacks of these systems are that they require the senior to have the device on his/her person and have it charged up. If the device is not charged or the senior forgets to carry it, the fall detection system will not function. For instance, the Apple Watch now offers a Fall Detection service. This wearable provides high sensitivity and specificity, but it also has some drawbacks. Seniors may forget to wear their Apple Watch or forget to recharge it. According to Apple, the Watch should be recharged every 18 hours \cite{RN1014}. Studies have shown this is a severe limitation to any practical implementation \cite{RN1014}.


\subsection{Radio-wave based}
Radio frequency signals have also been used to detect human poses, some using reflection/refraction and changes to the signal to infer location and pose \cite{Wang2017}. Others use RF tags placed at key points on a body to determine if a person has fallen \cite{Lustrek2009}. Finally, like the system described in this paper, there are implementations that use cameras and computer vision to analyze footage and try to pinpoint key locations on the body, either frame by frame, or over a series of frames, or through a keypoint heatmap, to determine if poses by people in the footage seem to imply that a fall has occurred \cite{RN1006}. Besides a large number of threshold-based systems, most of these implementations make use of machine learning to teach the system what a fall might look like \cite{RN1006}.  Radio-wave based human pose analysis is quite new and no fall detection systems have been implemented that use this approach \cite{RN1005}. The main limitation of this approach is that it is entirely based on the WiFi frequency in the living environment. Consequently, substantial on-site training is needed for the algorithm which may be prohibitive for wide-scale implementation in senior living environments and senior’s homes.


\subsection{Pressure-sensor based}

Pressure-sensor based fall detection systems detect the impact of the body with the ground or the near horizontal orientation of the faller following a fall \cite{fda2007}. Most fall-detection systems detect the shock received by the body upon impact using accelerometers \cite{RN997}. For example, Diaz et al. \cite{RN1034} developed a primary fall-detection system that involved a small adhesive sensor attached to the sacrum (the shield-shaped bony structure located at the base of the lumbar vertebrae). Its fall detection accuracy was 100\% with only 7.5\% of activities of daily living (ADL) predicted incorrectly as falls. Hwang et al. used a tri-axial accelerometer and gyroscope, both positioned on the chest, to differentiate between falls and ADL \cite{RN1019}. This system had a sensitivity of 95.5\% and specificity of 100\%. However, ADL testing of the system was only for three young adults who performed sitting and a daily life activity. To date fall-detection systems have used young participants to test the extent of misdetection of ADL as falls. Elderly people often move differently than younger people as they typically have less control over the speed of their body movements due to reduced muscle strength with old age.

Unfortunately, all pressure-sensor based systems have the same practical limitations. They require the senior to wear a patch or device that needs to be charged on a frequent basis (typically daily), is prone to damage and is unlikely to be worn on a consistent and continual basis \cite{fda2007, RN1012, RN997, RN1019}.


\subsection{Vision based}
Recent advancements in computer vision, machine learning, and deep learning, have provided a path forward to new and more accurate fall detection vision-based systems. Such systems are showing considerable success \cite{samira2018, RN1017}.  Machine Learning vision-based approaches for fall detection have been implemented by using training sets of different states a person can be in; namely, \textit{standing, walking, sitting, crouching, lying down} and other activities of daily living and states directly related to falling, namely, \textit{falling}, and \textit{fallen} \cite{RN1007, Wang2017, Lustrek2009}.  Fall detection machine learning algorithms have been implemented using Support Vector Machines (SVMs) \cite{Doukas2007, Homa2008, Liang2018}, kNN \cite{Jian2019}, and Naive Bayes \cite{Muneeswari2020}. The reported accuracy and sensitivity of these systems ranged from 82\% to 95\%.

However, like the other methods, there are some limitations in terms of practicality \cite{RN1018}. For instance, as most falls tend to occur in the bathroom \cite{AG2015}, it may not be appropriate to have video camera in this location \cite{RN1018}.


\subsection{Human Pose Estimation}
Human pose estimation is defined as the problem of determining the location of human joints (keypoints), such as knees, elbows, wrists, etc. in images or videos \cite{RN1003, RN1004}. It is also defined as the search for specific poses in the search space of all articulated poses \cite{RN1003, RN1004}. Human pose estimation is a new method attracting attention in the area of fall detection. When using specific cameras and computer vision for a fall detection system, pose data of a person can be used to train a fall detection ML system. Currently, there are two prominent pose detection algorithms available for researchers to create fall detection systems, OpenPose \cite{RN1003} and PoseNet \cite{posenet2019}. Fall detection systems have been created using OpenPose and have reported accuracies in the range of 85\%-96\% \cite{RN1020}. However, the computational requirement to run the detectors with OpenPose is substantial \cite{RN1003}. PoseNet is a much lighter weight algorithm that requires significantly less computational resources \cite{posenet2019}. It offers a comparable degree of accuracy at pose estimation; however, it can be run on simple low-powered devices (e.g., Arduino for instance).  This makes it attractive for practical implementations in the home or senior living environments.  At this time, there are no reported studies that have created a fall detection system using PoseNet. These considerations make PoseNet a promising approach worthy of consideration for this problem.