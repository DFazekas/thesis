\chapter{Introduction}
Elderly, 65 years and older, represent the fastest growing segment of the population worldwide \cite{AG2015}. In the United States, this was 13\% in 2010 and is expected to reach 20.2\% by 2050. In Europe, this was 22.5\% in 2005 and is expected to reach 30\% by 2050 \cite{RN1014}. Worldwide, the population of elderly individuals over 80 years old is over 140 million and is expected to more than triple that by 2050, increasing to nearly half billion \cite{AG2015}. The World Health Organization reports that nearly 1 out of 3 seniors will have a fall incident each year \cite{world2018}.  Around the world, falling incidences involving the elderly are quite common and detrimental to the health of those that have fallen \cite{fda2007}. Fall-related injuries currently cost the Canadian health care system \$2.8 billion per year \cite{RN1000}. Even a 20\% reduction in the rate of falls among Canadian seniors would translate into approximately 7,500 fewer hospitalizations and 1,800 fewer permanently disabled seniors, for an overall national savings of as much as \$138 billion per year \cite{RN1000}.



The first fall by a senior is usually the first fall of many to come, due to the fear of falling again and the gradually weakening bones of older human bodies \cite{RN1001}. These weaker bones make harsher injuries after a fall more likely, especially when much of the time when people fall – they are left alone without help for a period of time. It could take hours, or even a day, before someone discovers that this person has fallen when they live alone at home as many elderly citizens do \cite{RN1001}. Known as a \textit{long lie}, these long periods of laying on the floor waiting for support are likely to be terrible for health, pointing to more and more severe falls in the future \cite{fda2007, RN1011}.

In this paper, we address the problem of elderly falling and the \textit{long lie} by describing a computer vision machine learning (ML) and human pose analysis fall detection and alert system. The system is used to detect a fall happening in the area, and can send detection notifications securely to family members,  healthcare providers, and/or appropriate caregiver, immediately when the fall happens.

In this work, we present our fall detection system, specifically designed to be a lightweight application, unobtrusive, and easily deployed to many home or senior living facility environments. Furthermore, our system promotes senior independence as there is no reliance on the elder to use wearables whatsoever. While having devices on the person would likely increase the accuracy of a fall detection system, it does not help covering cases where the elderly might forget to wear a device, refuse to wear it, or not charge it before walking around their homes \cite{RN999}. The fall detection system presented in this paper meets these desirable characteristics through the use of camera footage, computer vision and machine learning. To our knowledge, there are no such systems that provide these features; this provides motivation and relevance for this research study. We validated our system against these desirable characteristics:


\begin{enumerate}
    \item The solution must be a lightweight application requiring low computational resources;
    \item Unobtrusively monitors the senior;
    \item Easily deployed to any home or senior living environment;
    \item No reliance on the resident to wear a device or wearable; and
    \item Must be secure and confidential.
\end{enumerate}


\section{Thesis Statement}
This is where you would state your thesis.  A thesis statement focuses your ideas into one or two sentences. It is crisp and to the point. 
A thesis statement describes the scope, purpose, and direction of the paper. It summarizes the conclusions that you have reached about the topic. 

\section{Outcomes and Contributions}
We addressed the desirable characteristics, designed, implemented and tested our system with young participants. The main contributions of this work are as follows.  Our system

\begin{enumerate}
\item is extremely lightweight; it can be run in a browser on any tablet, smartphone, or any low-powered device such as an Arduino, or Raspberry Pi;
\item discreetly monitors the senior in his/her home and requires no involvement from the senior;
\item can be easily deployed to any home or senior living environment;
\item has no reliance on the senior to wear a wearable in contrast to other systems that require a bracelet or pendant with a push button alarm to be worn;
\item is secure and confidential. All processing and decision making is done locally—only notifications indicating that a fall has been detected is transmitted to stakeholders; and
\item is accurate at detecting falls – 97\% accuracy across a diverse set of fall scenarios involving 9 participants.
 
\end{enumerate}
