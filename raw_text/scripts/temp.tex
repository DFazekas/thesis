\chapter{Introduction}
Emergency responders (ERs) are specialized, trained individuals operating emergency vehicles and arriving first at emergency scenes like road accidents or natural disasters. ERs typically include law enforcement officers (LEO), firefighters, and emergency medical service (EMS) technicians \cite{Hsiao2018}. Given the time-sensitive nature of emergencies, ERs need to quickly and safely reach their destinations \cite{Al-Ghamdi2002}. They thus are authorized to operate emergency vehicles with the Code Three Running option permitting the use of warning lights, sirens, exceeding speed limits, and crossing against stop signs and red lights to minimize collision risks and arrival times en route to their destination \cite{Hsiao2018, Al-Ghamdi2002, Buchenscheit2009}.

As the number of registered vehicles continues to grow each year, the increase in urban traffic volume leads to congestion and delays, resulting in increasing arrival times and risk of collisions for ERs when responding to calls \cite{Vlad2008}. ERs operate under stressful driving conditions, time pressure, and multitasking activities \cite{Hsiao2018}. For instance, the study by \cite{Buchenscheit2009} found that accidents involving emergency vehicles are four and eight times more likely to result in fatality and severe injury, respectively, compared to civilian drivers. The study also notes that the root cause for 30\% of these accidents stems from civilian drivers' wrong behaviour in avoiding the emergency vehicle. The relevant causative factors attributing to these crashes also include:

\begin{itemize}
	\item Complicated urban intersections \cite{Hsiao2018, Retting1995};
	\item High traffic volumes \cite{Hsiao2018, Vlad2008};
	\item Lack of recognition by other drivers \cite{Hsiao2018, Vlad2008, Sukru2020}; and
	\item Human error \cite{Buchenscheit2009, Sukru2020}.
\end{itemize}

In the United States, emergency vehicle-involved crashes account for thousands of injuries and hundreds of fatalities each year. Between 2004 and 2006, 37,600 LEOs were reported injured \cite{Hsiao2018}, 17,000 firefighters in 2015 \cite{Hsiao2018}, and 1,500 EMS technicians in 2009 \cite{Hsiao2018}. The average annual fatality count for ERs as a result of these road accidents is approximately 100 for LEOs \cite{Hsiao2018, Emergency-Vehicles2020, Enforcement-Safety2020}, 45 for EMS technicians \cite{Hsiao2018, Emergency-Vehicles2020}, and 15 for firefighters \cite{Hsiao2018, Emergency-Vehicles2020}. Furthermore, reports from \cite{Hsiao2018} show an average of 60 civilian fatalities each year due to these accidents, which also incur many lawsuits costing the cities millions of dollars due to these injuries, property damage, and life loss \cite{Hsiao2018}.


\section{Statement of the Problem}
The Ministry of Transportation (MOT) in Canada is dedicated to moving people safely, efficiently, and sustainably by promoting innovative technology and infrastructure \cite{MTO_2020}. When it comes to emergency vehicles, they are equipped with sirens and lights, and laws dictate how civilian drivers should respond when nearby an emergency vehicle \cite{MTO_2020}. For instance, failure to slow down and make room when approaching an emergency vehicle or failure to maintain at least 150-meters behind a travelling emergency vehicle will result in fines between \$400 to \$2,000 and three demerit points in Ontario based on Section 159(2) and (3) of the Highway Traffic Act \cite{MoveOver_2021, MTO_2020}. 

Unfortunately, the traditional methods used by emergency vehicles (i.e., lights and sirens) are human perception-based and have been proven ineffective at attaining the attention of civilian drivers and negotiating traffic in urban areas \cite{Buchenscheit2009, Missikpode2018}. A driver must multitask while operating a vehicle, giving attention to the movements of surrounding vehicles, nearby pedestrians, and any other threats that could emerge. By the time civilian drivers acknowledge the warning signals emitted by emergency vehicles, not only do they have difficulty identifying the direction and distance of the source, they may not have sufficient space or time to make the best maneuver, thereby providing minimal aid in reducing arrival times and collision risks for emergency vehicles \cite{McDonald2013, Buchenscheit2009}.

Continuing with the reliance on human perception-based warning signals results in additional traffic chaos and accidents \cite{McDonald2013, Missikpode2018}. Developing a more sophisticated and assistive system for emergency and civilian drivers could help the MOT increase road safety in urban areas by minimizing the flow rate achieved by strategically distributing traffic flow from primary roads to secondary roads. 

A significant number of studies have focused on further developing Intelligent Transporation Systems (ITS), focusing on increasing road safety \cite{Rizvi2007}. One of the common goals among the studies was to reduce the chance of emergency vehicle-involved accidents while also reducing their arrival time. Many papers used approaches to strategically control traffic lights to block traffic from entering an emergency path. Other approaches used telecommunication technologies such as DSRC embedded in the vehicle to provide drivers with a system for earlier warning notifications and maneuvering suggestions \cite{Huang2009}. Many papers show that a promising approach is to reduce the number of vehicles on the path of an emergency vehicle by taking alternative routes (e.g., secondary roads or side streets that would typically result in slower commutes but contain less traffic volume) \cite{Huang2009}. Research has shown that offloading the driver's avoidance decision-making to a centralized server significantly reduces the traffic volume and delays incurred by the emergency vehicles \cite{Huang2009}.

\section{Purpose of the Study}
This study will take these ideas of rerouting civilian drivers off the immediate path of emergency vehicles and take them further. The research will simulate various traffic scenarios in urban areas with varying ratios between regular vehicles and IoT-enabled vehicles, referred to as connected vehicles (CVs). The CVs will use cellular data to establish near-real-time two-way communication with our cloud server as it provides traffic data and receives navigation directions to avoid route collisions.

We define a route collision as the event of a travelling civilian vehicle getting closer than the safety threshold of 150-meters, enforced by the MOT \cite{MoveOver_2021, MTO_2020}, to a travelling emergency vehicle. For the system to avoid route collisions, drivers of both types of CVs (i.e., an emergency vehicle and civilian vehicle) need to enter their destinations before driving and follow the paths provided by our system. This study explores the relationship between intelligently regulating traffic flow between primary and secondary roads and the arrival times for emergency vehicles within urban areas such as Toronto, Ontario.


\section{Outcomes \& Contributions}
The study by \cite{Huang2009} uses a centralized traffic control server to improve road safety and reduce arrival times for emergency vehicles. The server notifies civilian drivers with a message to pull over and halt until the approaching emergence vehicles pass. The server also generates the emergency vehicle's optimal routes using near-real-time traffic information collected from roadside units (RSUs) installed at every intersection. This approach relies too heavily on high penetration rates of RSUs in urban areas' infrastructure, which would be an expensive endeavour to install and maintain, posing scalability, security, and adoption challenges \cite{Tonguz2020}. Additionally, research shows that warning drivers of approaching emergency vehicles without providing maneuver suggestions result in panicking the driver as they are left to evaluate the situation themselves, increasing the risk of further accidents with the emergency vehicle or neighbouring civilian drivers \cite{Buchenscheit2009, Sukru2020}.

This research will focus on preemptively assisting civilian drivers in avoiding route collisions and congested road segments. Our Cloud-Based Road-Traffic Risk Mediator System will guide CVs to maintain the safety threshold of 150-meters \cite{MoveOver_2021, MTO_2020} from all emergency vehicles during their commute. Our system uses predictive analytics and path arrangement algorithms to suggest an optimal detour to the civilian driver of CVs when it predicts that they will collide with an emergency vehicle somewhere along their commute. This cognitive offloading of the decision-making aims to create a stress-free situation by providing ample time for drivers to safely exit the primary path minutes before regular drivers would acknowledge the traditional warning signals of emergency vehicles. As a result, fewer vehicles would be on the emergency path, providing more space for regular drivers to quickly pull over and halt. 

This research aims to provide the MOT with a cost-effective improvement to the existing emergency warning system that vehicle manufacturers could mandate on newer models. The system would embed as standard within their IoT-enabled motor vehicles, helping drivers make better decisions that result in safer, smoother, and faster commutes.


\section{Research Question}
This study explores near-real-time route guidance for CVs in regulating traffic volume and flows in urban areas. This study's factors include the headway, V/C ratio, and arrival times. The populations that this study will explore are civilian drivers and emergency responders in regular and CVs in urban areas, such as Toronto, Ontario. 
This study aims to answer the research question: Does the use of CVs and autonomously mediating urban traffic flow from primary to secondary roads improve road safety and reduce arrival times for emergency vehicles?


\section{Significance of the Study}
By leveraging cloud computing and IoT technologies, civilian drivers will overcome human perception limitations at identifying emergency vehicles, more effectively avoid emergency vehicles, and emergency responders will have fewer vehicles in their path to maneuver around when responding to emergency calls. All of this contributes to minimizing the risk of emergency vehicle-involved accidents, reducing arrival times to emergency sites, and creating safer and smoother commutes for road users.


\section{Overview of Methodology}
Our experiments are simulation-based, consisting of simulation software and a cloud-hosted server. Each experiment tests various realistic traffic situations designed to evaluate the effectiveness of our system against varying penetration rates of CVs. When our server generates a civilian CV's path, it searches for possible route collisions with emergency vehicles within the city. It uses the respective vehicle's current location, speed and traffic data to estimate whether the vehicles could collide (i.e., appear within the 150-meters \cite{MoveOver_2021, MTO_2020} threshold of each other). If a route collision is likely, we modify the civilian driver's path, optimizing it for the fastest commute while avoiding the collision areas.


\section{Organization of the Thesis}
This chapter introduced our topic and the problem we will be studying. We looked at why we need to study this and how we will benefit from it. In chapter 2, we explore the review of related literature. Chapter 3 incorporates the methodology that we are going to use to conduct our experiments. Chapter 4 reports on our findings. Chapter 5 provides our conclusions and recommendations.
